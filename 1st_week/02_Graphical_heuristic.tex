Básicamente lo que plantea este video es que si queremos mostrar datos de una manera gráfica muchas veces "menos" es "mas".
A esto se refieren con el término Graphical Heuristics (GH). En este contexto, heuristico se refiere a una regla o proceso que te ayude y sirva como guía a la hora de tomar una decision.
Es más bien una solución práctica que dista de ser perfecta u óptima, pero que te guia en tu camino a encontrar la solución final más óptima.
Edward Tufle trabaja con este término y pone como ejemplos 2 tipos de GH:
a) Data-ink ratio
b) Chart junk


a) Data-ink ratio se define como la cantidad de tinta que usas para mostrar los datos versus la cantidad total de tinta que usas para mostrar todo el gráfico.
La ides es que ese ratio sea lo más cercano a 1, es decir que uses la menor cantidad de tinta en adornar un gráfico. De esta manera, la información que se muestra queda mucho más simple de leer
y se evita mostrar información irrelevante o redundante.

b) Otra forma de optimizar los gráficos es a través del Chart junk. Existen 3 tipos de Chart junk:
   i) Unintended optical art (Unintended optical art): un exceso de de sombras y patrones en un gráfico puede arruinar lo que se quiere mostrar. En vez de usar diseños con patrones, es mejor etiquetar ls información.
   ii) The grid: Muchas veces una malla muy gruesa puede causar confusión a la hora de leer los datos. SI decidimos usar una grid es mejor que sea delgada, para no saturar con líneas un gráfico.
   iii) The duck: (Gráficos creativos sin datos). Inclusión de fotografías en el gráfico. Generalmente se usan en periódicos o revistas. Un gráfico muy conocido se llama Diamonds Were a Girl's Best Friend, de Nigel Holmes. EL grafico representa la tendencia del precio de diamantes entre 1978 y 1982. En el se ve la figura de una mujer sentada con las piernas cruzadas en donde la curva del precio de diamantes sigue la forma de la pierna de la mujer.

Existen otras categorías para optimizar un gráfico "heuristicamente":
- Data words: Edward Tufte enfatiza mucho la idea del minimalismo y simplicidad en los gráficos. El argumenta que un pequeño gráfico, por ejemplo uno de línea de series de tiempo, puede transmitir información mucho más rápido para el lector. El llama a este item sparklines.
              Esta idea es muy usada por Microsoft Excel. En el ejemplo muestran una tabla con datos de 4 compañías y sus ganancias a lo largo de un período de tiempo. Al final de cada columna con datos podemos ver un pequeño gráfico donde se muestra con una simple línea como evolucionan las ganancias de cada compañía dando un muy buen marco general de los datos.
- Lie Factor: Es el tamaño de un efecto mostrado en el gráfico dividido por el tamaño del efecto en los datos.
              Muchas veces pequeñas diferencias se pueden mostrar de manera poco clara mediante representaciones visuales, llevando a confusión al lector.

Alberto Cairo describe en su libro 5 caracteristicas escenciales para una buena visualizacion de gráficos.
Es muy importante recordar que a pesar de mencionar y definir cada una de estas caracteristicas por separado, en realidad ellas estan interconectadas entre si.

a) Veracidad: La persona que trabaja con datos debe ser honesta a la hora de limpiar, manipular y resumir los datos. Debemos considerar explicitamente cada modificacion que aplicamos a estos.
              Ademas hay que ser muy escéptico con los resultados, ya que al estar indagando en la busqueda de patrones, muchas veces se puede perder la objetividad, llegando a conclusiones equivocadas.

b) Funcionalidad: Un gráfico con muchos adornos e información redundante carece de funcionalidad a la hora de extraer la informacion. Como dijimos antes "menos" es "más" cuando queremos mostrar información importante.
c) Belleza: El concepto de belleza es bastante amplio y depende de varios factores como por ejemplo la audiencia a la cual va dirigido el gráfico. El género y la cultura de la audiencial, etc.
d) Perspicacia: Un gráfico debe ser capaz de no simplemente mostrar la información de una forma más entendible, sino que también debe ser capaz de crear un momento de "eureka" al lector.
e) Iluminador: Esta caracteristica se compone de las 4 antes mencionadas más una dimensión extra de responsabilidad ética social.
