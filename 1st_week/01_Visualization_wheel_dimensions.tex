Alberto Cairo proporciona una forma de entender como diseñar herramientas de visuaizacion basado en una estructura llamada la rueda de visualizacio.
En este esquema tenemos 2 polos importantes, que se desglosan en sub items
a) Visualizacion mas compleja y profunda.   | b) Visualizacion mas superficial y facil de entender.
    i) Abatraccion (Tablas y esquemas)      |      vii) Figuracion (dibujos y figuras)
    ii) Funcionalidad (sin adornos)         |      viii) Decoracion (embellecer)
    iii) Densidad (estudios mas profundos)  |      ix) Ligero (analisis simples y rapidos)
    iv) Multidimensionalidad                |      xi) Unidimensional
    v) Original                             |      x) Familiar
    vi) Novedad                             |      xii) Rudimentario, redundancia


Modulo de visualizacion de imagenes
Structure of a Supercell
https://mooctools.ai.umich.edu/multimeasure/424-visualization-wheel-practice/67-structure-of-a-supercell/a434c91b833eac54843b60d8dcb6f15e0b6ce92c/
